\documentclass[10pt]{article}


\usepackage[english]{babel}
\usepackage[utf8]{inputenc}

%
% For alternative styles, see the biblatex manual:
% http://mirrors.ctan.org/macros/latex/contrib/biblatex/doc/biblatex.pdf
%
% The 'verbose' family of styles produces full citations in footnotes, 
% with and a variety of options for ibidem abbreviations.
%
\usepackage{csquotes}

\usepackage{lipsum} % for dummy text
\setlength{\parskip}{1em}
\setlength{\parindent}{0cm}


\begin{document}
\title{On realism,
on its effect on natural language
and their combined effect on creating Abstract Objects}
\author{Yehoshaphat Schellekens}
\maketitle

\begin{abstract}
\par
This paper has to say a little about realism, and a bit more about its effect on  language. While its main intention is to present a process that explains how humans create abstract objects. \par
The theory bellow, will demonstrate that  the nature of abstract objects (such as numbers, or other adjectives such as \textit{beauty}) is very different than the common known explanations, which suggests that its abstractness depends on whether it exists , or that abstract is the opposite of \textit{concrete}, for example a belief that \textit{beauty}  is abstract because it isn't real, while a \textit{This beautiful woman} is real. \par
In contrast to this explanation, I will demonstrate that what makes a objects an abstract object is independent of how that object relates to reality but rather how the part of speech that was used to express it relates to reality, abstract isn’t a feature of numbers or  beauty, its about sentences and nouns. \par
Drilling into how sentences and nouns interact with reality, might result in the common practice  that nouns tend to represent objects in reality , and sentences represent  thoughts upon objects, I will later on claim that this belief results from a higher level  perception of all humans, which is known as \textit{realism}. however, this practice (that nouns represent objects and   sentences  represent thoughts) is problematic. \par
Nouns, as will be demonstrated, have desired properties which sentences lack, and, due to the “belief” stated above, it's complicated to exploit these virtues when it comes to thoughts, as they are expected to be expressed in sentences. \par
The following paper will demonstrate that an abstract object is a \textbf{\textit{subjective opinion wrapped up in a linguistic form that resemble a noun}}, and in practice is an alternative form to a sentence. This practice is essentially a \textbf{\textit{workaround}} that solves the problem which was forced open us humans due to are realistic perception of reality and its effect on language.
\end{abstract}

\newpage


\section*{Preface}
Past work on this topic is summarized In here \cite{standford} , A major claim that is constantly  brought up is the distinction between Abstract Objects vs Concrete objects, which also serves at some extent as ways to try to define what makes an object, an Abstract Object. Among these efforts the “Non-Spatiality Criterion” and “The Causal Inefficacy Criterion”(site)  were suggested, which can interpreted as a property that describes how it relates to reality. Two other finding that might result from this type of explanation will likely to relate abstract objects to mathematics or to generalization (a belief that 2 is more general than 2 specific apples, and in general, a belief that abstract objects describe groups)\par

Indeed, I think that my theory gives a clearer way to classify a object as abstract,  but there is yet another important reasoning to my rejection from the current explanation. The concrete abstract distinction does not explain two highly important questions that still remain mysterious: \par
\textbf{\textit{Why do abstract objects play such useful and important role in our lives even if they resemble entities that do not exist?}}\par
Abstract objects have played for years crucial rolls in both ordinary life's and also in a larger extent in life's of professionals, especially in domains that makes extensive use of math, such as engineering and statistics, for their practitioners mathematical objects are far from fairy tails.
\par
Notice here that I emphasis that  abstract forms play a dominate role in professionals life. Generally, people learned to use them in an academic institute, and in many cases report that these studies where rough, people find these topics highly unintuitive which leads me to another topic that I believe my theory can handle:\par

\textbf{\textit{What is exactly in abstract objects that makes them so hard to grasp and use?}}
\par
Indeed it is possible to claim that the reason for the difficultly is tied to the fact that people find it hard to imagine objects they cant visualize, but here my main criticism lies in the fact that this kind of explanation do not offer any ways for students to ease this difficulty.\par
The theory bellow will present a methodology that can both answer these questions, and explains why people relate abstract objects to imagination, to generalization or to math and would also involve several examples.
\newpage 
\section*{On the great virtues of the noun}

When creating a proper syntactic sentence, the noun has a role that can resemble a building block on which complex linguistic structures are based, such as verbs and adjectives. Following is a list of features that nouns posses,  these features combined makes the noun a powerful tool, these features, or shall I say virtues results mainly from the strong assertion of nouns as words that represent real objects.

A Noun is:
\begin{enumerate}
\item
\textbf{ Believed to express a real thing in reality:} in many cases a noun is believed to represent a single object, or several  objects, that exists in reality , this claim is followed by another claim, that this relation goes the other way around, and expressing some thing as a noun implies that it really exists.
\item
\textbf{It has a short from, that is believed to be bounded} nouns have a tendency of being short, and in many cases when they are long, they are shortened by abbreviation, also nouns, just like objects, are believed to have a clear boundary that separates them from the words that surrounds them.

\item
\textbf{ The representation of the noun in reality is believed to be indifferent to its form} A form  that represent that noun has no affect on its representation, and is chosen in an arbitrary manner ( a cat wouldn't be affected if ill change its naming from \textit{cat} to  \textit{gat}). This feature further enables another desired feature \textit{“ease of translation”}, its relatively easy to translate nouns from one language to another. Again, this indifference results from the natural assertion of the noun to reality, as a key knowledge of reality is that its indifferent to what people say about it.

\item
\textbf{ Highly used in dictionaries (and in some extent encyclopedia)} the features stated above all contribute to the fact that dictionaries make extensive use of nouns. Dictionaries have a structure of a collection of noun and meaning pairs. A noun that is presented in a well known dictionary will establish the feeling that its meaning is well established as well, and usually reflects a convention by many individuals. The use of dictionary, allows another important benefit, it allows several people, such as researchers to independently study the meaning of that noun, or things that are related to it, for example, two researchers can run independent study on lions, and add at separate times results of their study to the same  entry to that noun within the encyclopedia.

\item
\textbf{It has a standalone form}  Forms like adjectives or verbs require nouns to be binded to them whenever they are in use,  this feature gives sentences with nouns a feeling that it can be represented as a form of function and argument  (\textit{“x was here yesterday”}, \textit{“x is smart”},...). Nouns however, gives a feeling that the are standalone, its easy to imagine a noun without a verb, the opposite is less intuitive.

\par
Before ill state the next benefit of a noun, ill start to roll an explanation that will be elaborated further on in the following sections, regarding an essential step required, when creating sentences. Sentences require additional effort apart from being a collection of words, they require, an effort, which I felt that the best naming for it, is  \textbf{\textit{intent}}* which is an action that is intended to make the sentence feel more realistic for an individual or to his surrounding. A good example of intent, is when a person express a sentence and along side the expression he will imagine in his mind descriptive images of what that sentence represent, that gives the sentence the feeling that its real.

*\textit{In order to makes thing easy, from now on ill refer to intent as an action that takes place in a context of sentences, even though it could be expressed in other context as well.}


\item
\textbf{The ease of Intention}
when using sentences that have both nouns, and other speech components, the focus of the intention will be the other Non noun language components .And  whenever a noun is in use within the context of the sentence, the user of that noun can significantly lose his intention regarding what that noun means relaxing the need for additional effort (like  creating descriptive images in his mind),  For example, when I'm expressing , \textit{“lions are fierce animals”} my intention will focus on imagining fierceness of animals and binding it to lions, but will not focus on the meaning of lions.
\end{enumerate}

\par
All these virtues, and especially the ease of intention, makes it easy to create what I like to call anonymous sentence, a sentence where the nouns are replaced by some place holder / variable, for example \textit{“X is fat”} , \textit{“X is on top of Y”}.
\par
This Summery serves as an introduction for another form that also tries to make use of these virtues, sentence, which are our main tool for expressing thoughts which is believed to be either true or false.

\newpage 
\section*{What sentences are made of}
When it comes to understand how abstract objects are created its important to analyze first their competitors, sentences, and specifically focusing on  a major drawback they possess, their requirement for intention. To better understand this deficit its best to see a sentence as comprised out of 3 components, a \textit{thought} associated at some way to a real \textit{object} warped together with \textit{intent}.\par
The simplest form of a sentence that highlights this decomposition  is of the form : \textit{“Sam is ugly”}, \textit{“lions are scary”} ( as opposed to \textit{“lions are brown”} which is probably an indication for more objective situation). It has a component that’s  related to an object, that is a noun (\textit{“Sam”}), a component that relates to a belief regarding that noun (\textit{“ugly”}) and it has another vague component \textit{“is”} that I urge to be related to the last necessary ingredient of a sentence, the “intent” component. At some extent all forms of sentences  have a structure that tries to capture these 3 elements.
\par
So how does intent takes place and why is is so important?
\par
Intent is an action that takes place along the technical expression of a sentence, and its requirement is that it will give the sentence (an expression that is targeted towards thoughts) looks and feels of describing a real thing, as it would have describing a real object. In general, intent is a collection of words, notations, imaginations and actions that a person who express a sentence caries out to himself or to others in order to highlight that the thought that the  is expressing has a similar nature to the noun he is using within it, meaning that the sentence is part of reality.  I believe that this effort is a prerequisite in order to make the sentence a true sentence, Truth is usually enabled in reality not in our imagination.
\par
Following are examples of how intent can take place:\par
\begin{enumerate}
\item
\textbf{\textit{This, that: “This ball is red”, “That tree it tall”} ( while pointing them out to others  in reality)}, these are probably the simplest examples of what I refer to as intent in sentences, words (along with actions) that highlights to the surrounding that I am talking about reality by pointing out to specific objects.
\item
\textbf{Capital letters / dots} Unlike nouns, sentences have designated notations that are intended to give them looks an feels of nouns, and nouns as was explained previously have  a feature of being bounded from their surrounding, capital letters and dots ,as tools for highlighting the sentence from its surrounding, are a good example of some extra effort  required to make sure that the sentence has a nounish looks, hence it feels real.
\item
\textbf{Imaginary images} Just as described at previous sections, intent could be implemented in some imaginary description in ones mind of what that sentence feel. A good example are sentences that includes verbs, as they could be images of some movement accruing over time.
\item
\textbf{the word \textit{“is” }} At first glance, is sounds like the most unnecessary component within a sentence,  why is it so hard to express sentences (especially with adjectives) without it? My answer to that, is that \textit{is}, the intent component, is there to express that the user wants to convince that the thought component within the sentence has a similar nature to the noun component, they are both describing real things. \textit{Is} can be interpreted as some sort of equality and within a sentence the most  important equality, is that the nature of the noun will be equivalent to the nature of the thought.
\item
\textbf{highlighting word} Highlighting the beginning and ending of a sentence,  can give a similar bounding effect, as described in 2. Another example which applied in Hebrew (And in some extent in English) is a case where the “is” component is neglected when a sentence is highly highlighted (\textit{“Dans smart!!”}) my intuition for this type of intention, is that highlighting implies that the information within the sentence has deeply affected the creator of the sentence, and only things that  happen in reality can really get to your mind.
\item
\textbf{The noun itself} True that the noun, also has a informative role, but apart from that it has a role of helping on caring out the intention of the sentence, a good example of a sentence that uses a noun just for intent is an anonymous sentence, in such a sentence the variable plays only as a utility that enables intent \textit{“X is smart”}, is a sentence, but “is smart” isn't.
\end{enumerate}
\par
The need for extra effort, due to intent, makes sentences  highly demanding forms to express, a good demonstration of that is some difficulty humans have in creating conjunctions for sentences, it is hard to describe relations between sentences, harder than describing  relations of two nouns within a sentence.\par

The reason for all this effort results from the fact that sentences focus on the subjective world, since the counterpart world, I.e. the objective world, does not need any words to become apparent,it had has every human sense to its aid. Human senses includes vision, sound, smell and feeling, which are all believed to be far more powerful than words. We use words, when the objective world is out of reach for the speaker, his surrounding or in cases when the speaker isn’t even referring to the objective world, but to whats in his mind. As a result of this inferior-ness we use sentences in order to emphasize that the words we express are referring to reality, a task which is executed by a sentence using intention, which in one hand allows the thought to become real but doing so leave no room to achieve other goals, especially when it comes to using the resulted sentence within yet another demanding follow up sentence, a difficulty which will be described in the following section.\par

\subsection*{Intent is intended to make the sentence feel real, not True}
Intent isn’t about convincing the surrounding whether the sentence is a True sentence, but rather to give a realistic feeling, for example when stating that “Dan is smart” I'm clearly suggests that this sentence deals with reality even though Dan might turn out to be a complete idiot. Further more, I claim that fictitious stories that are bundled into a sentences will give the feeling that they deal with reality , since this is the exact feeling sentences are intended to achieve.\par
A good example of this distinction that involves using logical operators \textit{(“or” , “and”,..)}  can be seen in a sentence of the form \textit{“Dan is both smart and interesting”} , in this kind of sentence,  the elements that required  to relate it to reality  (“Dan is”),  stay the same, but another component ( \textit{“and interesting” }) that is added to a simple form of a sentence ads complexity to the final \textit{“thruth-ness”} of that sentence. This is added, easily without any effort. On the other hand notice the following phrase that might be true though sounds weird: \textit{“I love eating humus and united states is independent since 1776”}, true that a possible explanation for the weird feeling comes from our desire to talk in context, but this kind of weird conjunction, has a deeper root – the difficulty to create two different intentions  both at the same time \par

The last element that fuels the cycle of nouns, sentences and intent, is our realist perception of the world, which relates true statements to reality, hence forcing the forms that are intended to eventually express the truth to become tools that as a start convince the surrounding that they properly represent reality.\par
Realism comes in cost, lets explore in depth, what exactly we are paying to keep it.

\newpage 
\section*{Sentences dream of becoming nouns}

When ever there’s use of a sentence, one would want to have the same capabilities as nouns allow, a great example of this is a case where one expresses a thought (using a sentence) which he will  want  for a start to condense  into a short form, later on be used in yet other sentence and even  embed it into a dictionary. unfortunately this capability isn't so easy when it comes to sentences, as they require intent. Lets illustrate this difficulty in the following (hopefully amusing) example.\par
I would like to express my opinion regarding \textit{Sam}, a well known man. My opinion contrast against the  convention that Sam is a fine looking human being, while I think as him as an ugly person, further more, I believe that  my thought is ground breaking, and in the future people would recognize my vision and how looking forward it was. As a starting point I express my though to any one that is willing to here me that \textit{Sam is ugly}, it seems to work, people are listing to my ideas. Now its time to leverage this success.\par
A good indicator of my success, will be that  other reaserchers will use this discovery for further investigation, and even better, it could be shown up as a new entry in Wikipedia, where a huge gratitude towards my discovery would be shown.\par
So how will the researcher further use my original discovery, for follow-up discoveries, and how will express it?  he will probably write some thing like, \textit{Sam's ugliness is caused due to his thick eyebrows}.
\par
Why did the researcher change the original format of my thought from ‘Sam is ugly’ to “Sam’s ugliness”, why cant he just say that \textit{Sam is ugly  is caused due to his thick eyebrows} leaving my original sentence in its original form?
\par 
An initial explanation for the conversion is that by stating that ‘Sam’s ugliness was...” the researcher confirms he is agreeing with the statement, however if he wanted to do that he would just have said “Sam is indeed Ugly”, also if he wanted to show his agreement as a side note he would have expressed “Sam the ugly was ...” , my claim is that this transformation has a completely different nature, the need to create a follow up sentence on top of the original sentence, and In order to achieve that , he had to dress it into a form of a noun, which enabled him to use all the virtues of the noun, while avoiding the need to intent the original sentence. Doing so, our editor needs to create an abstract form, which technically can be seen as  a transformation of a sentence into a Fraze. 
\par
The need to intent a sentence, in-order to make it feel real is a major drawback, for one to further use that sentence in a follow up sentence, since the action of intention is demanding and require concentration, and do not allow the user to focus his intention on the higher level sentence.\par
\par
A good example that shows in which cases its actually easy to embed a sentence within another  is a sentence of the type \textit{i think that  Dan is a smart kid},  in this case the higher level sentence is specifically expressing that it has nothing to do with reality, but just focused on a thought, this action avoids the need to intent the higher level sentence, there fore allowing the bottom level sentence form to remain unchanged.
\par
So far I've illustrated the “damage” realism brings, though fortunately realism isn’t the only way to view our world. Humans have a remarkable capability of viewing the world in an opposite way, and  imagine it as  of thoughts that belong to the  independent world that the “outside” world is dependent upon, a perception that is known to be called solipsism, let us see how this perception (or shall I call it capability ) is exploited to create a workaround for our realist problem.

\newpage 
\section*{Exploiting solipsism}
The need to make further use of the sentence, requires  for non standard actions, even ones that will neglect our need to express real thoughts, a typical situation is where the need to express sentences about thoughts is more important that to understand how these thoughts relate to reality\par
Such a move is creating an abstract object, an action that suppresses the root cause of our need for intention, our realism perception. Such suppression is done through creating in our mind, for a short period of time (the very same time we need to use our original sentence ) a counter-factual reality where objects depends on thoughts. This is done for a practical reason, a need to create a robust, lightweight ,and non demanding  from that can express content that is usually wrapped as a sentence, but looks like a noun, to be further used within another higher level sentence, which in many cases is the real focus of interest. Unlike sentence intention where the user recruits reality that includes external effort to give it the nourish feeling, abstractness, the creation of abstract objects views reality as if it was second order citizen just under the privileged thoughts, hence deteriorating the original need to create that intention.\par

\subsection*{Solipsism capability}
Solipsism, as a well known philosophical idea is usually considered a hack. people  stick to realism as the absolute truth, and don't really see solipsism other that a  trick. Solipsism might be hack, but when it comes to creating abstract objects its the working horse tool that creates it, a person that cant imagine reality as if the real world was Dependant on his thoughts wont be able to create abstract objects.
Just one last remark to distinct between realism and solipsism, when it comes to creating abstract objects, the creator prefers to keep his realist belief and only uses solipsism for a very short period. Recall that its a belief that going against it would actually consider a person, an insane one. This is why In this context I refer to realism and solipsism as \textit{realism belief} vs. \textit{solipsism capability}.\par
The exploitation of solipsism for a short time give an understanding why people find it hard to understand and use abstract forms, since the analysis of the nature of abstract forms and how to use them, is usually conducted after the abstract creation has end, through the view of a realist, ill further elaborate on this topic later on.\par
Up until now I've given the reader all the required background to understand the processes , its time to demonstrate it, in several examples bellow.\par

\newpage 
\section*{Examples}
\subsection*{the suffix - ness}
\textit{A native English suffix attached to adjectives and participles,forming abstract nouns denoting quality and state (and often, by extension, some thing exemplifying a quality or state):
darkness; goodness; kindness; obligingness; preparedness.
(from: http://www.dictionary.com/browse/ness)}
\par
An easy way to understand the process of creating abstract objects, is by first creating a sentence, and then transitioning it into such a form, and as previously explained, the easiest sentence to start with is of the form \textit{“x is y”}, where y represents a thought, usually as an adjective.\par
Lets return to our Sam example and see how solipsist capability kicks in. Our abstract creator first express \textit{Sam is ugly} , then for a short moment undermines his realist belief and embraces a solipsist alternative, for now the world of objects depends on thoughts, and the world of thoughts is the world of truths.\par

At this point, the thought, “ugly”  benefits the privilege and capabilities that used to belong to the noun, especially the ease of intent, for now the ugly component is playing the role that was previously saved to Sam. This solipsist point of view enables to transform it into \textit{Sam's ugliness}, a form which highlights that ugliness is the stand alone component, that Sam is dependent on. This transformation gives the benefit of becoming a building block in a new  follow up sentence. In this new form, "Sams ugliness”, ugliness inherits the benefits of being a noun alike form, if the user want to talk about the ugliness that is related to Sam, he can keep doing so, but he doesn't have to rely on that noun for the need of intention. For now the noun plays only the role of speaking about Sam, and the other role which it had in a sentence, that is to enable intent, becomes irrelevant, hence making it easy to drop out.\par

Adjectives that are transformed using the ness suffix are probably the most wide spread abstract objects people interact in their daily life's, this results from the fact that they were generated from a sentence that all 3 ingredients (noun, thought and intention) are clear and separate, and its clear that the sentence represent a thought, but what about sentences that try to capture reality in an objective manner, for example \textit{Dan ran}, allegedly describing reality and not a thought? \par
Lets recall that a sentence will emerge, only in cases were reality haven't made its full reach. If both viewers saw that Dan indeed ran  and were convinced that their companion observed the event as they did, they wouldn't have bother saying anything, this is why I consider this sentence, or any other type of sentence as a thought. Adding to that,verbs just don’t feel like  stand alone forms, and a sentence with a verb is still hard to use within a follow up sentence,  “Dan ran was impressive” just doesn't sounds right.\par

After elaborating that the sentence \textit{Dan ran} is a thought that requires intent, its time to see how we can get rid of it, and “Dans run” which can be further used in  \textit{Dans run was impressive} does the job, again using the solipsism capability needed for the conversions, and just like before the noun (Dan)  in the abstract from “Dans run” can easily be removed.\par
Before ill move on to the next example, ill start to leverage the explanations and example so far and explain why abstract objects are tied to other concepts such as  generalization and nonexistence.\par

\paragraph{Abstract vs Generalization and vs Non Existence}
The conversion of the form “Sam’s Ugliness” to just “Ugliness” is enabled through  solipsism capability and creates the feeling that “Sam’s ugliness” is a specific member of the group of ugly things. Although there is some logic in binding these concepts, I choose to refrain from this explanation for several reasons. At start, the reason for creating the abstract object wasn't intended to remove the noun component, but to remove the intent component, Sam's Ugliness” is just as abstract as “Ugliness”, Second of all, generalization is a far easier task than abstractation, there is no need to invoke any solipsism when grouping several objects such in  \textit{“this cat and that dog are both animals}”, and for last, I don’t think that  groups (especially of objects) create any feeling of components that aren't part of reality.\par
Speaking of things that feel unreal, Abstract forms are created in a context which switches rolls between what is real and what is not, which results in a form that its probably impossible to tell whether it represents some thing in reality.  its not the specific form that its realness isn’t clear, its the whole environment that surrounds its creation that is problematic. 


\subsection*{Numbers}
Going away from the simplest type of sentence, one that is  composed of thoughts about nouns , is a slightly harder from of a sentence, one that is comprised from a noun and its  properties. A good intuition of a property, is to see it as a type of description that is not essential to the existence of the object, and in some extent can bee seen as a point of view of that noun. When looking for an ideal sentence as a starting point to get to a native number (integers, 1 to 9), this kind of sentence satisfies the requirement. \par

For a start, its important to emphasize that numbers don’t always play a role as abstract objects. For example, the following sentences, \textit{This is a single apple}, or in a different formulation  \textit{This apple is represented by one}  are both sentences that describe the object \textit{apple} and its property (\textit{single}, \textit{represented by 1}).\par
A number will become an abstract object only if the sentence that caries it is transformed (through solipsism capability) to a form that emphasis that the object is dependent on thought, for example \textit{singleness on an apple} or \textit{representation of one, for an apple}”, which now  makes it possible to  make claims about the representation of an apple: \par
 \textit{This representation of one for a  apple, is similar to that representation of one of an orange”.}
 \par
and again, since the orange and the apple aren't essential  for intention, and the representations are playing a role of stand alone components, its possible to convert this sentence into the well known mathematical notation \textit{1=1}. 

\paragraph{Different types of numbers correspond to different types of sentences} 
When taking an imaginary walk in the number line, there seem to be “magic jumps”, from 9 to 10, from positive to 0 and then to negative numbers, or from real to imaginary. Since abstract objects play a role as alternative forms to sentences, I believe that the reason for these jumps results from the desire to capture different types of sentences, for example sentences with verbs, or sentences with negation.\par
Negative number, as I believe, correspond to sentences with verbs,  \textit{A single apple was removed}, which can be  transformed (along with dropping the noun)  into the abstract object \textit{removal of a one}, which is equivalent to the known notation “-1”, and just like in any other abstract object, the need to express a sentence on top of the abstract object (example: “A single number that is added to -1 is equal to 0” ) is more important than the need to understand what -1 represents.\par
The number 0, as I believe, corresponds to sentences with negation, “A banana, is not a number”, which can be transformed into (along with dropping the noun) the abstract object “not -ness of a number”, which is equivalent to the known notation “0”, and again, the creator of the abstract object should not worry at all regarding what does “not -ness of a number” represent, the main trade off that creating abstract object is to loss the capability of  representation in exchange of  gaining capability of using it in a follow up sentence.

Numbers aren't the only abstract objects that mathematician’s exploit, and in fact professional (surprisingly)  hardly encounter any of them in their daily work  . Solipsist capability has created many mathematical objects like functions and derivatives. lets explore one of these advanced abstract objects in the following example.

\subsection*{ Multi dimensional point}

 In very similar way to a number, a point  can play dual role as either abstract object or as a component of a sentence (this time paling the role of the noun). When looking at the sentence \textit{a point is represented by  both x and y labels} the point isn’t playing a role of abstract object.\par
However if this sentence is transformed into \textit{x and y representation of a point}, we can get closer to the mathematical object point, which is playing a role as an abstract object, and there isn’t any prevention from us to , the Fraze \textit{x, y and z representation of a point}. Mathematics views a point through its properties, unlike the standard point which is considered an object which is represented by a noun.\par
Another interesting object that is created is a similar way is the imagery number, also known as \textit{i}, this object can be viewed as transforming the sentence \textit{A square root was applied on minus 1} , into \textit{applying  square root  on minus 1} (which is later converted into a shorted form of \textit{i})\par
And just like any other abstract object, the destruction of the tool that enables to understand if its part of reality, that is the sentence, prevents us from understanding if these two abstract objects are part of reality. Don’t worry about that, mathematicians aren't worried from that as well, understanding if some thing isn’t real is the job of the philosophy department.

\newpage 
\section*{Why is it so hard to deal with abstract objects?}

A common explanation of the difficulty to handle Abstract Objects ties the difficulty to the fact that they are hard to grasp, many abstract objects don’t have a clear representation in reality. This specific explanation should be first elaborated, its not that abstract objects are imaginary or real its that the processes that is used to create them makes reality and imagination mixed up.\par
This initial explanation has indeed some impact, but there are other two explanations which this theory has to offer, the first, which plays a minor role, is that creating abstract objects requires one to go near the madness zone, a zone that undermines a very fundamental belief of how we belief reality takes place, the other explanation, which plays a major role, is that abstract objects wear two hats, one of an image that would normally be described as a sentence, and on othe other hand make looks and feels of a noun, making it hard to understand how to \textbf{use} it.\par
The first  example of difficulty, is the different way of how mathematicians see a point, through its properties (which is usually tied to use it in a sentence) from an ordinary point, which is seen an an object (which should be used as a noun).\par
Another example, which I personally encountered, while teaching statistics and probability, relates to  explaining the nature of a random variable ( usually normally distributed random variable).\par
A reasonable explanation of a random variable, for example \textit{The salary of an American employe}”, can be explained in within a framework of a sentence:\par
 \textit{There is 10 percent chance of being 4000 or 20 percent of being 5000, or …….}, but a student will usually encounter it in a label that usually represent a noun like X, or the well visual of a random variable as surface (like the famous bell curve surface), which are both perceived as objects that are usually reflected as a noun.\par
Dual entities as such, are every where in math, whether its random variables, multi dimensional objects, limits , derivatives, or imaginary numbers, a math student will constantly deal with entities that are treated in class an nouns, but reflect situations that should be pronounced in sentences, and this difficulty I believe, is a major draw back for students approaching studies that make extensive use of abstract objects.\par 
Hopefully understanding the dual nature of abstract objects, especially emphasizing that they are actually sentences, bundled as nouns,  may defiantly help while teaching these studies to students.\par

\begin{thebibliography}{9}
 
\bibitem{standford}
Stanford Encyclopedia of Philosophy 
Gideon Rosen
\textit{Abstract Objects}
\\\texttt{https://plato.stanford.edu/entries/abstract-objects/}. 

\end{thebibliography}

\end{document}